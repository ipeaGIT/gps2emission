%% Copernicus Publications Manuscript Preparation Template for LaTeX Submissions
%% ---------------------------------
%% This template should be used for copernicus.cls
%% The class file and some style files are bundled in the Copernicus Latex Package, which can be downloaded from the different journal webpages.
%% For further assistance please contact Copernicus Publications at: production@copernicus.org
%% https://publications.copernicus.org/for_authors/manuscript_preparation.html

%% copernicus_rticles_template (flag for rticles template detection - do not remove!)

%% Please use the following documentclass and journal abbreviations for discussion papers and final revised papers.

%% 2-column papers and discussion papers
\documentclass[gc, manuscript]{copernicus}



%% Journal abbreviations (please use the same for discussion papers and final revised papers)


% Advances in Geosciences (adgeo)
% Advances in Radio Science (ars)
% Advances in Science and Research (asr)
% Advances in Statistical Climatology, Meteorology and Oceanography (ascmo)
% Annales Geophysicae (angeo)
% Archives Animal Breeding (aab)
% ASTRA Proceedings (ap)
% Atmospheric Chemistry and Physics (acp)
% Atmospheric Measurement Techniques (amt)
% Biogeosciences (bg)
% Climate of the Past (cp)
% DEUQUA Special Publications (deuquasp)
% Drinking Water Engineering and Science (dwes)
% Earth Surface Dynamics (esurf)
% Earth System Dynamics (esd)
% Earth System Science Data (essd)
% E&G Quaternary Science Journal (egqsj)
% European Journal of Mineralogy (ejm)
% Fossil Record (fr)
% Geochronology (gchron)
% Geographica Helvetica (gh)
% Geoscience Communication (gc)
% Geoscientific Instrumentation, Methods and Data Systems (gi)
% Geoscientific Model Development (gmd)
% History of Geo- and Space Sciences (hgss)
% Hydrology and Earth System Sciences (hess)
% Journal of Bone and Joint Infection (jbji)
% Journal of Micropalaeontology (jm)
% Journal of Sensors and Sensor Systems (jsss)
% Magnetic Resonance (mr)
% Mechanical Sciences (ms)
% Natural Hazards and Earth System Sciences (nhess)
% Nonlinear Processes in Geophysics (npg)
% Ocean Science (os)
% Polarforschung - Journal of the German Society for Polar Research (polf)
% Primate Biology (pb)
% Proceedings of the International Association of Hydrological Sciences (piahs)
% Scientific Drilling (sd)
% SOIL (soil)
% Solid Earth (se)
% The Cryosphere (tc)
% Weather and Climate Dynamics (wcd)
% Web Ecology (we)
% Wind Energy Science (wes)

%% Please DO NOT add additional packages or LaTeX commands to the template. They
%% are not supported by Coperncius. LaTeX packages already
%% included in the copernicus.cls are:
%\usepackage[german, english]{babel}
%\usepackage{tabularx}
%\usepackage{cancel}
%\usepackage{multirow}
%\usepackage{supertabular}
%\usepackage{algorithmic}
%\usepackage{algorithm}
%\usepackage{amsthm}
%\usepackage{float}
%\usepackage{subfig}
%\usepackage{rotating}

% Pandoc citation processing

% The "Technical instructions for LaTex" by Copernicus require _not_ to insert any additional packages.
%

\begin{document}

\title{GTFS2EMIS an R package to estimate public transport emissions
based on GTFS data}


\Author[1, *]{Rafael}{H.M. Pereira}
\Author[2]{Pedro}{R. Andrade}
\Author[1, *]{Joao Pedro}{Bazzo}


\affil[1]{Institute for Applied Economic Research, Brasília, Brazil}
\affil[2]{National Institute of Space Research, São José dos Campos,
Brazil}

\runningtitle{R Markdown Template for Copernicus}

\runningauthor{Nüst et al.}


\correspondence{Joao Pedro\ Bazzo\ (joao.bazzo@gmail.com)}



\received{}
\pubdiscuss{} %% only important for two-stage journals
\revised{}
\accepted{}
\published{}

%% These dates will be inserted by Copernicus Publications during the typesetting process.


\firstpage{1}

\maketitle


\begin{abstract}
The abstract goes here. It can also be on \emph{multiple lines}.
\end{abstract}


\copyrightstatement{The author's copyright for this publication is
transferred to institution/company.}


\section{Introduction}

Transport emissions have been widely recognized among the leading and
growing contributors to global emissions \citep[
]{caiazzo_air_2013, nocera_assessing_2018}. There is also scant evidence
of how transportation activities impact air quality in cities
\citep{landrigan_lancet_2018}, aggravating the negative short- and
long-term effects pollution has on children premature deaths
\citep{currie_traffic_2009}, cardiovascular diseases
\citep{brook_particulate_2010, turner_long-term_2016}, ischemic stroke
\citep{wellenius_ambient_2012} and cognitive development
\citep{chen_living_2017, fu_association_2019, shehab_effects_2019, zhang_impact_2018}.
This has a higher impact in low and middle-income countries
\citep{combes_fine_2019}, especially in more vulnerable groups, such as
elderly people \citep{yap_particulate_2019} and children
\citep{braga_health_2001, gauderman_association_2015}. Understanding the
spatial and temporal patterns of emissions is a key factor to adopt more
precise public policies to reduce air pollution exposure
\citep{clark_national_2014, targino_spatial_2018}, improve the housing
and land use planning, and subsidize public investments Particularly
with public transport policies, extensive efforts have been made to
quantify air pollution patterns and to understand the impact of fleet
technology, fuel, topography, and route choices on overall levels of air
pollution (refs).

Most of the studies on public transportation emissions focus on
emissions monitoring with remote sensing devices (refs), onboard
diagnosis using PEMS (refs), vehicle counts data (refs), radar
equipments (refs), public transport GPS data (refs), and fuel based
estimates (top down approaches). However, due to a restriction of
spatial application, or costly local data collection, some of these
approaches might not be easy to scale for other cities. Other
approaches, such as air quality analysis (refs), satellite data
observations (refs) would represent the final state of air quality but
ultimately do not look only at bus emissions. Studies that identify and
map pollution in low and middle-income countries are still needed,
particularly when easily reproduced.

Our analysis contributes to the public transport emission literature by
developing a novel methodology that leverages a widespread open data
format for public transport networks known as General Transit Feed
Specification (GTFS) to estimate vehicle emissions in a high spatial and
temporal resolution. The method is presented in a R package
``gtfs2emis'' to allow friendly and reproducible use for different
applications. In this paper, we present estimates on CO2, NOx, PM10, CO,
CH4 hot exhaust emissions for 24 cities in the world, considering a
typical business day on October 2019 --- as a baseline date. A
comparison of the vehicle emissions rates of those transport networks is
also presented, along with a discussion on the versatility and
limitations of the method.

The remainder of this paper is organized as follows. Section two
presents a brief literature review on public transportation emissions.
Section three describes the data and methods for the public transport
data, emission factors databases and vehicle emissions estimates..
Section four presents the results, looking at the overall levels of
pollutant emissions and its spatial and temporal resolution behavior.
Finally, section five presents the final remarks and discusses some
environmental insights that can be drawn from the package.

\section{Methods}

\subsection{Transport model}

\subsubsection{GTFS data}

The method used in the paper leverages a global standard format for
public transport data known as General Transit Feed Specification
(GTFS). The GTFS data of a given public transport system is collected in
a ZIP file, that brings detailed geolocated information on scheduled
services including its stops, routes, trips, stop times and calendar
organized in a structured text file.

This format was originally created in the mid-2000s by Google and
Portland transport authority and it has since then become adopted
worldwide by more than 600 cities
\citep{noauthor_openmobilitydata_nodate}. The simplicity of this data
format and its wide adoption creates a common ground for researchers and
practitioners across the globe to develop and share computational
methods that can be seamlessly deployed across multiple cities to manage
fleet allocation, plan transport services, etc. Detailed aspects of the
GTFS contents are found in \citep{noauthor_reference_nodate}.

We expand the sample of cities with GTFS data collected directly from
local transport authorities and from Transit Feed archives - the largest
GTFS open data repository website. The GTFS data were pre-processed to
filter by bus routes and to check consistencies on the required files
(e.g.~agency, stops, routes, trips, stop times). The months of October
and November 2019 are used as the baseline date for the emissions
estimates, in order to avoid possible differences in public
transportation schedules due to the COVID-19 pandemic. A summary of the
main GTFS information is described in Table 1, together with the fleet
data.

\subsection{GTFS2gps}

The process to convert GTFS files to GPS-like spatial data points is
done for every trip of the schedule. It interpolates the space-time
position of each vehicle in each trip considering the network distance
and average speed between stops. It samples the timestamp of each
vehicle every 50m.

Once a GPS data format is generated from GTFS, a few adjustments on
speed is made. Since the mean speed is estimated in a street link
between two bus stops (e.g.~stop ``i'' and ``i+1''), the data on
departure time of ``i'' and arrival time of ``i+1'' is necessary. If
``i+1'' arrival time is very close or the same as the departure time of
``i'', the speed will be very high (\textgreater100 kph) or infinite. In
such cases, we consider the speed to be the mean speed of all the trips
in this specific bus route. Therefore, the departure times of the GPS
are recalculated based on the new travelled speeds. This adjustment is
also necessary in street links without valid arrival or departure times:
before the first bus stop of the route (where only arrival time is
available); after the last bus stop (only departure time is available);
and trips with invalid stop times.

\subsection{Emission model}

\subsubsection{Emission factors}

Emission factors (EFs) are empirical functional relations between
pollutant emissions and the activity that causes them (Franco et al.,
2013). This study uses EF based on Environmental Sanitation Technology
Company (CETESB) --- for Brazil; California Air Resources Board (CARB)
--- for the California U.S; Motor Vehicle Emission Model (MOVES) for
regions in U.S. outside California; European Environment Agency (EEA)
--- for European countries. Among these sources, EMFAC, MOVES and
EMEP/EEA have speed-dependent emission factors. For the Brazilian
cities' estimates, the local EF is scaled to the EMEP/EEA function
considering the average speed of the vehicle driving cycle. The
brazilian emission factor is related to speed by the expression

\begin{equation}
EF^{scaled}_{i,j,k,l}=EF^{local}_{i,j,k,l}\frac{EF(V_i)_{j,k,l}}{EF(V_{dc})_{j,k,l}}.  
\end{equation}

where \(EF_{i,j,k,l}^{scaled}\) is the scaled emission factors for each
street link \(i\), bus type \(j\), fuel \(k\), age \(l\),;
\(EF_{i,j,k,l}^{local}\) is the local emission factor;
\(EF(V_i)_{j,k,m}\) and \(EF(V_{dc})_{j,k,l}\) is the EEA emission
factor at the speed of \(V_i\);and the driving cycle \(V_{dc}\) of
FTP-25 (33 kph).

The CARB's (California Air Resources Board) EMFAC (EMission FACtor)
model estimates statewide and regional emissions inventory by
multiplying emissions rates with vehicle activity data. The running
exhaust emission factors are distributed according to speed bins from 5
to 90, with 5 mph incremental. It allows modeling one season (summer,
winter or annual average) due to temporal variations of EF due to
meteorological conditions (temperature and relative humidity); and
geographic area (Statewide, Air Basin, Air District, Metropolitan
Planning Organization, Country and Sub-Area). Detailed data on emission
factors are found in \citet{noauthor_emfac_nodate}.

\begin{table*}[t]
\caption{Emission factor sources, related bus categories and variables associated.}
\begin{tabular}{ccc}
\hline
Source & Buses categories & Variables \\ \hline
CETESB & Micro, Standard, Articulated & Age, Fuel, EURO stage \\
EMFAC model & Urban Buses & Age, Fuel \\
EMEP/EEA & Micro, Standard, Articulated & Fuel, EURO stage, technology, load, slope \\
MOVES U.S EPA & Urban Buses & Age, Fuel \\
\hline
\end{tabular}
\belowtable{Table footnotes}
\end{table*}

\begin{table*}[t]
\caption{Association between model year, PROCONVE Norm and EURO stage for the brazilian fleet (adapted from ICCT).}
\begin{tabular}{ccc}
\hline
PROCONVE Norm & EURO equivalent & Year range \\ \hline
P4  & II  & 1998 - 2003 \\
P5  & III  & 2004 - 2011 \\
P8  & V & 2012 - 2019\\
\hline
\end{tabular}
\belowtable{Table footnotes}
\end{table*}

\begin{table*}[t]
\caption{Summary of all vehicle classes covered by the Tier 2 methodology}
\begin{tabular}{ccc}
\hline
Urban bus category  & Euro Stage  \\ \hline
Urban CNG buses & Euro I, Euro II, Euro III, EEV   \\
Urban buses   &  Conventional, Euro I - Euro VI\\
Urban Diesel Hybrid  & Euro VI  \\
Urban biodiesel buses   & Conventional, Euro I - Euro VI\\
\hline
\end{tabular}
\belowtable{Table footnotes}
\end{table*}

European exhaust emission factors are presented in the EMEP/EEA air
pollutant inventory guidebook, considering the COPERT 5.4 Software
version. This study considered average load on all trips and zero slope
rate for all studied cities. Urban Buses are presented in three main
categories, which are Urban Buses Midi (\textless= 15 t), Standard (15 -
18 t), Articulated (\textgreater18 t), Urban Diesel Buses Hybrid, CNG
(Compressed Natural Gas) Buses. The speed dependent emissions factors
for diesel urban buses have been taken from HBEFA (Handbook Emission
Factors for Road Transport), for Euro I to Euro VI emissions standards.
Distinct parameters of EF were considered for Euro V standard, according
to the control technology, that can be Exhaust Gas Recirculation --- EGR
or Selective Catalytic Reduction --- SCR. According to EMEP/EEA, it is
estimated that 75\% of Euro V heavy-duty vehicles are equipped with SCR.
For the category of CNG buses, it has an additional emission standard
known as Enhanced Environmental Vehicles (EEV), since it may have
different combustion, after-treatment technology, and are associated
with lower PM and NOx emission rates compared with diesel buses (add
source). Only older CNG buses are classified in EURO I, II, or III.
Detailed aspects of hot exhaust emissions rates of Europe are found in
the EMEP/EEA guidebook.

\subsubsection{Emission estimates}

The estimates of hot exhaust emissions for each trip is given by

\begin{equation}
EH_{i,j,k,l} = L_i\times EF_{i,j,k,l}
\end{equation}

where \(EH_{i,j,k,l}\) is the emission for the street link \(i\),
vehicle category \(j\), fuel \(k\) and age \(l\); \(L_i\) is the length
of the street link \(i\)(km); \(EF_{i,j,k,l}\) is the emission factor
(g/km). In order to evaluate emissions spatially, the overall emissions
are allocated into a 350m resolution grid (H3 resolution from Uber).

\subsubsection{Emission post-processing}

Subsubsection text here.

\section{Estimate Curitiba's emissions using GTFS2emis model}

\subsection{Traffic data for Curitiba}

\subsection{Emission factors}

\subsection{Emission estimates}

\conclusions[Discussions and Conclusions]



\codedataavailability{use this to add a statement when having data sets
and software code
available} %% use this section when having data sets and software code available

\sampleavailability{use this section when having geoscientific samples
available} %% use this section when having geoscientific samples available

\videosupplement{use this section when having video supplements
available} %% use this section when having geoscientific samples available

%%%%%%%%%%%%%%%%%%%%%%%%%%%%%%%%%%%%%%%%%%
%% optional

%%%%%%%%%%%%%%%%%%%%%%%%%%%%%%%%%%%%%%%%%%
\appendix
\section{Figures and tables in appendices}

Regarding figures and tables in appendices, the following two options
are possible depending on your general handling of figures and tables in
the manuscript environment:

\subsection{Option 1}

If you sorted all figures and tables into the sections of the text,
please also sort the appendix figures and appendix tables into the
respective appendix sections. They will be correctly named
automatically.

\subsection{Option 2}

If you put all figures after the reference list, please insert appendix
tables and figures after the normal tables and figures.

To rename them correctly to A1, A2, etc., please add the following
commands in front of them: \texttt{\textbackslash{}appendixfigures}
needs to be added in front of appendix figures
\texttt{\textbackslash{}appendixtables} needs to be added in front of
appendix tables

Please add \texttt{\textbackslash{}clearpage} between each table and/or
figure. Further guidelines on figures and tables can be found below.
\noappendix

%%%%%%%%%%%%%%%%%%%%%%%%%%%%%%%%%%%%%%%%%%
\authorcontribution{Daniel wrote the package. Josiah thought about
poterry. Markus filled in for a second author.} %% optional section

%%%%%%%%%%%%%%%%%%%%%%%%%%%%%%%%%%%%%%%%%%
\competinginterests{The authors declare no competing
interests.} %% this section is mandatory even if you declare that no competing interests are present

%%%%%%%%%%%%%%%%%%%%%%%%%%%%%%%%%%%%%%%%%%
\disclaimer{We like Copernicus.} %% optional section

%%%%%%%%%%%%%%%%%%%%%%%%%%%%%%%%%%%%%%%%%%
\begin{acknowledgements}
Thanks to the rticles contributors!
\end{acknowledgements}

%% REFERENCES
%% DN: pre-configured to BibTeX for rticles

%% The reference list is compiled as follows:
%%
%% \begin{thebibliography}{}
%%
%% \bibitem[AUTHOR(YEAR)]{LABEL1}
%% REFERENCE 1
%%
%% \bibitem[AUTHOR(YEAR)]{LABEL2}
%% REFERENCE 2
%%
%% \end{thebibliography}

%% Since the Copernicus LaTeX package includes the BibTeX style file copernicus.bst,
%% authors experienced with BibTeX only have to include the following two lines:
%%
\bibliographystyle{copernicus}
\bibliography{global-pt-emission.bib}
%%
%% URLs and DOIs can be entered in your BibTeX file as:
%%
%% URL = {http://www.xyz.org/~jones/idx_g.htm}
%% DOI = {10.5194/xyz}


%% LITERATURE CITATIONS
%%
%% command                        & example result
%% \citet{jones90}|               & Jones et al. (1990)
%% \citep{jones90}|               & (Jones et al., 1990)
%% \citep{jones90,jones93}|       & (Jones et al., 1990, 1993)
%% \citep[p.~32]{jones90}|        & (Jones et al., 1990, p.~32)
%% \citep[e.g.,][]{jones90}|      & (e.g., Jones et al., 1990)
%% \citep[e.g.,][p.~32]{jones90}| & (e.g., Jones et al., 1990, p.~32)
%% \citeauthor{jones90}|          & Jones et al.
%% \citeyear{jones90}|            & 1990

\end{document}
